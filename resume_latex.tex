
\documentclass{article}
\usepackage{titlesec}
\usepackage{titling}
\usepackage[papersize={8.5in,11in},margin=1in]{geometry}

\hyphenpenalty=10000

\pagenumbering{gobble}
\titleformat{\section}
{\large\bfseries} {}
{0em}
{}[\titlerule]

\titleformat{\subsection}
{\normalsize\bfseries}
{$\bullet$}
{0em}
{}

\titleformat{\subsubsection}[runin]
{\small\bfseries}
{}
{0em}
{}

\titlespacing{\subsubsection}
{0em}{0.5em}{1em}

\titlespacing{\subsection}
{0em}{0.75em}{0em}

\titlespacing{\section}
{0em}{0.8em}{0.25em}

\renewcommand{\maketitle}{
\begin{center}
{\Large\bfseries\theauthor}
\vspace{.25em}

\begin{center}
\begin{tabular}{ r l c r l }
 \textbf{email} & raymond@sutrisno.me & --- & \textbf{phone} & 909 706 1288 \\
 \textbf{github} & tetrovolt & --- & \textbf{linkedin} & in/tetrovolt
\end{tabular}
\end{center}

\end{center}
}

\title{R\'esum\'e}
\author{Raymond A. Sutrisno}
\date{}

\begin{document}
\maketitle

\section{Education}
\subsection{University of Houston (2016 - 2020) \textbf{GPA}{3.94}}
\textbf{Major}: Bachelors of Computer Science, \textbf{Minor}: Mathematics

\section{Skills}
\begin{tabular}{p{5cm} l}
    \textbf{Front-End Technologies} & HTML, CSS \\
    \textbf{Back-End Technologies} & NodeJS, Express, Postgresql \\
    \textbf{Machine Learning} & ScikitLearn, Keras, Tensorflow, numpy, pandas, OpenCV \\
    \textbf{Programming Languages} & Python, Java, C++, JavaScript  \\
    \textbf{Tools} & bash, git, make, linux/unix, \LaTeX, matlab
\end{tabular}

\section{Projects Experience}
\subsection{Reamer IP Bot\textbf{(Spring 2018)}}
\noindent{$\ \bullet$} Discord bot used for WAN ip retrieval for house.

\noindent{$\ \bullet$} Used to circumvent needing a static ip business plan from internet provider.

\noindent{$\ \bullet$} Works as both a discord bot with command and a http server with JSON api.

\noindent{$\ \bullet$} \textbf{project-link:} github.com/TetroVolt/ReamerHouseBot

\subsection{Java Ray Tracer \textbf{(Spring 2018)}}
\noindent{$\ \bullet$} Ray tracer rendering demo written Java.

\noindent{$\ \bullet$} Main goals were to simulate refraction from transparent objects and reflection geometry.

\noindent{$\ \bullet$} \textbf{project-link:} github.com/TetroVolt/Java-RayTracer

\subsection{Expressions Evaluator}
\noindent Simple Mathematical Expression Evaluator written in Java. Written without
regex for tokenization. Uses reverse polish notation.

\noindent{$\ \bullet$}\textbf{project-link:} "github.com/TetroVolt/Expressions

\section{Job Experience}
\begin{flushleft}
\begin{tabular}{p{12.5cm} p{2.5cm}}
{$\bullet$}\textbf{Teaching Assistantships, \textit{University of Houston}}

    $\ \ \bullet$\textbf{Advanced Machine Learning} COSC 7462 (for Graduates)(Spring 2019)

    $\ \ \bullet$\textbf{Introduction to Computer Science} COSC 1306 (for Undergrads)(Fall 2018)

    $\ \ \bullet$\textbf{Artificial Intelligence} COSC 4368 (for Undergrads)(Spring 2018)

    $\ \ \bullet$\textbf{Machine Learning} COSC 6342 (for Graduates)(Fall 2017)
    & \textbf{Fall 2018, Spring 2018, Fall\ \ \ \ \ \ \ 2017} \\[0.75cm]


{$\bullet$}\textbf{National Science Foundation Research Assistant Internship, \textit{University of Houston}}

    $\ \ \bullet$ Topic: Image Classification of Dewetting Microscopy

    $\ \ \bullet$ Implemented image processing techniques enhance and extract features
    from microscopy images of polymer dewetting. 

    $\ \ \bullet$ Techniques formalized into preprocessing pipeline for machine learning.

    $\ \ \bullet$ Trained and tested various models such as SVM, Neural Networks, Random Forest.

    $\ \ \bullet$ \textbf{project-link}: github.com/gtoti/Summer2018REU

    & \textbf{Summer 2018}
    \\[0.75cm]


{$\bullet$}\textbf{Research Assistant, \textit{University of Houston}}

{$\ \bullet$}Paper: R. Vilalta, \textbf{R. Sutrisno}, E. E. O. Ishida, R. Beck, R. S. de Souza, A. Mahabal,
``\textit{Photometric Redshift Estimation: An Active Learning Approach}'' IEEE SSCI 2017.

    $\ \ \bullet$ Tasked with implementing "Query By Committee" active machine learning algorithms from research papers for galaxy distance estimation using photometric astronomical data.

    & \textbf{Summer 2017}


\end{tabular}
\end{flushleft}


\section{Publications}
\noindent \textbullet R. Vilalta, \textbf{R. Sutrisno}, E. E. O. Ishida, R. Beck, R. S. de Souza, A. Mahabal:
``\textit{Photometric Redshift Estimation: An Active Learning Approach}'' IEEE SSCI 2017


\section{Awards}
\subsection{University of Houston Deans List}
\subsection{HP CodeWars 2016 3rd Place, \textit{Hewlett Packard} March 2016}
HP Code Wars Computer Science Competition in Houston, TX

\end{document}

